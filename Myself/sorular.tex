// -- Birkaç cümleden oluşan bir hikaye yazıp myStory isimli bir değişkene atayalım. İçerisinde tekrar eden birkaç kelime bulunsun.
// 1- Yazdığınız hikayenin uzunluğunu yazdırın.
// 2- Hikayenin toplam index sayısı kaçtır.
// 3- string içerisinde tekrar eden bir kelimenin ilkinin indexini ve sonuncusunun indexini sorgulayın.
// 4- İlk 15 karakterini yazdırın. ( 2 farklı yöntem kullanarak)
// 5- 15. karakterden sonrasını yazdırın. ( 2 farklı yöntem kullanarak)
// 6- Son 5 karakteri yazdırın. ( 2 farklı yöntem kullanarak)
// 7- 11. karakterden sonra gelen 10 karakteri yazdırın.
// 8- Son 5 karakter haric hikayenizi yazdırın.
// 9- Hikayeniz, seçeceğiniz bir kelimeyi içeriyor mu diye kontrol edin.
// 10- Hikayenizdeki tüm "i" karakterlerini "ı" ya çevirin.
// 11- Hikayenizdeki "a" karakterini "e" ye çevirin.
// 12- Bütün metni büyük harfe çevirin.
// 13- Bütün metni küçük harfe çevirin.
// -----------------------------------------------
// 14- prompt üzerinden kullancıdan bir isim isteyin ve bir değişkene atayın.
//   a- Ardından bu değişkeni kullanarak console'a "Hoşgeldin, isim" şeklinde yazdırın.
//   b- Yukarıdaki çıktının aynısını template litteral kullanarak konsola yazdırın. Örn: console.log(degisken) => "benim adım: Cem"  
// 15- Kullanıcıdan gelen stringin başına ve sonuna boşluk koyun. 
//   a- Başındaki boşlukları kaldırın.
//   b- Sonundaki boşlukları kaldırın.
//   c- Boşluksuz halini yeni bir değişkene atayarak konsola yazdırın.
// 16- Çıktıyı uygun metotları kullanarak "BENİM Adım: İsim" yazacak hale getirin.
// 17- Ekrana tırnak içindeki mesajı ya da değişkendeki değeri yazdıran bir pop-up oluşturun. "Ara vakti. İyi dinlenmeler"
